\documentclass[12pt,titlepage]{article}

\usepackage{fancyhdr}
\usepackage{graphicx}
\usepackage{lastpage}
\usepackage[body={6.0in,8.0in}, left=1in, right=1in, top=1in, bottom=1in]{geometry}

\newcommand{\blankpage}{\hfill\thispagestyle{empty}\pagebreak\addtocounter{page}{-1}}

\pagestyle{fancy}
\lhead{Teaching Boxes Builder: Design Brief}
\rhead{\today}

\title{Teaching Boxes Builder: \\ Design Brief}
\author{Deanna Fierman  \\ \small{fierman@colorado.edu} \and
        Cory Maccarrone \\ \small{Cory.Maccarrone@colorado.edu} \and
		W. Evan Sheehan \\ \small{Wallace.Sheehan@gmail.com}}

\begin{document}
\maketitle

\blankpage

\pagenumbering{roman}
\tableofcontents
\cfoot{\hrule \thepage}
\pagebreak

\blankpage

\pagenumbering{arabic}
\cfoot{\hrule Page\ \thepage\ of\ \pageref{LastPage}}

\section{Problem}
The teaching boxes project intends to provide teachers with a digital resource
for lesson plans and materials. Our project's purpose is to design a user
interface for teachers to create and modify lesson plans in this digital
library.

\section{Methods}
% The design process for this project begins with user interviews. From the
% interviews we generate tasks that are used to create a low-fidelity prototype.
% After the low-fidelity prototype is evaluated, we create a high-fidelity
% prototype. The high-fidelity prototype is evaluated by people outside of
% the development team. These evaluations provide feedback to improve the design
% of the final product, which is developed after the evaluation of the
% high-fidelity prototype. Finally, once the final product is ready, it is
% evaluated by some target users.

\subsection{Progress}
As of October 19, 2005, we have interviewed 4 users. Each interview lasted about
thirty minutes.  Currently, we are trying to schedule two more interviews.
According to our schedule, we should have already finished interviewing and
started on our low-fidelity prototype. We will have to push the due date for the
low-fidelity prototype back a week. And rather than having a week to evaluate
the prototype after it is finished, we can evaluate it while we develop it so
that the low-fi prototype is completed and evaluated by October $27^{th}$. Then
we will be back on schedule.

\subsection{Interview Questions}
\begin{itemize}
	\item How do you create lesson plans? Is it a structured process, or
		different every time?
	\item How do you revise lesson plans, and how frequently?
	\item How long does it usually take you to create a lesson plan?
	\item Where do you get ideas for lesson plans?
	\item How do you find resources for lesson plans?
	\item Do you work collaboratively with other teachers?
		\begin{itemize}
			\item At what point in the process do you collaborate? During?
				After?
			\item How does this process work?
		\end{itemize}
	\item What's your biggest concern about using someone else's lesson plans?
	\item What do you think you might like about using someone eles's lesson
		plans?
	\item What is a lesson plan to you?
\end{itemize}

\section{Interviews}
\subsection{KK -- October 6, 2005}
\paragraph{Profile} KK taught high school science for deaf students. She taught
for two years in Boulder county.

\paragraph{Interview}The first thing KK does when creating a lesson plan is to
identify unit goals from the curriculum. She uses resources like text books to
develop each lesson. The amount she uses the resources depends on how familiar
the lesson is to her; more familiar lessons require fewer resources during
development. She also spends time going over labs, activities, and her own
lessons developed previously. KK looks for ``essential questions'': important
questions that should be answered for the students by the end of the lesson or
unit. Her lessons usually start with a journaling question to start the students
thinking about the lesson. As a science teacher, she did a lot of hands-on
exercises in her classes.

KK worked for a time with a group of teachers as part of a professional
development program where they did their lesson planning together. In these
lesson planning sessions they would each work on their own lesson plans
individually, but because they were in a group, they were able to bounce ideas
off of each other and give suggestions. Although she feels that this type of
group planning is not very common, KK felt that this arrangement worked
well.

Because all classes are different, KK revises her lesson plans with each new
class to better suite them. This usually involves adding things to the lesson,
rewriting lab methodology, or updating a lesson for new equipment. She might
change the reading level or questions to better match the class. KK shared labs
with other teachers more often then she shared lesson plans. She also shared
concepts for lessons with other teachers. Sometimes, she would look on-line for
lesson plans, but never used them unmodified.

In her mind, a lesson plan consists of an objective, a layout, questions,
examples, and some form of conclusion. The objective comes from unit goals and
the curriculum. A layout contains notes about the lesson, a schedule, etc. And
the conclusion of the lesson should tie the lesson into the unit somehow.

\subsection{GF -- October 7, 2005}
\paragraph{Profile} GF taught middle school math.

\paragraph{Interview}GF begins creating a lesson by breaking down the year into
units, and then breaking down the units. This helps her to identify the context
of each lesson within the unit. Using the context, GF defines a goal for the
lesson. She creates a list of activities. The degree of detail in the activity
depends on the newness of the activity: more familiar activities require less
detail.  Because she didn't have to submit formal lesson plans, she just kept a
notebook of her plans.

Most of the collaborative planning she did was casual. GF would discuss her
class with other teachers, and they would share ideas. They would sometimes
share lessons that were developed individually, and make modifications to them.
She always planned her year out before it started, and her units before the
start of each one. She reviewed each of her old lessons before reusing them,
sometimes making changes or annotating them.

GF considers a lesson plan to be a context, a goal, activities, a wrap-up, and
an assessment. The context defines the lesson's relationship to the rest of the
unit. The wrap-up supplies a means to connect the lesson to the rest or the unit
and provide closure for the students.

\subsection{SZ -- October 7, 2005}
\paragraph{Profile} SZ taught high school English for eight years in Illinois.
She taught a range of classes there from introductory classes for freshmen to
senior-level Advanced Placement classes, and is currently teaching a course for
undergraduates at CU.

\paragraph{Interview}SZ plans lessons in a different manner than she does a most
of her other work.  She is dedicated to using her computer except when she's
planning lessons; she prefers to write the lesson plans out by hand.  SZ finds
it difficult to write a lesson plan without having an overall plan for the topic
covered and a goal for the whole course.  Therefore, she begins with an overall
course goal, narrows the goal down into units, and then decides what specific
lessons would be helpful. She usually designs specific lessons the day before
the class, and writes step-by-step what she plans to do in class. SZ mostly
plans lessons alone, in a variety of locales, although she had team-taught
before.

She primarily receives her resources from other teachers, although she cannot
teach another person's lesson plan verbatim.  She says she must "do something to
make it my own," edit it somehow to fit her needs for the lesson and goals for
the course, and make notes to herself on that lesson plan.  If she's taught a
particular lesson or course before, she commonly goes back to her notebook after
class and makes notes to herself about what worked and what needed to be
changed.  Often things are eliminated due to "hokiness," lack of time, because
she doesn't like it, or it doesn't quite fit the goals, material, or students.
SZ isn't shy about borrowing from other teachers who have done interesting
things in the classroom, and commonly actively searches for interesting
activities other teachers are using.  Other source that she uses when searching
for resources for lesson planning include her own notebooks, colleagues in the
department, textbooks, research she has done for her master's thesis which can
be integrated into the classroom, academic journals, and her own experiences as
a student.

SZ did not have a typical lesson layout; the order of events changed depending
on the material they were going to cover.  She did, however, specify that a good
lesson includes objectives, materials, procedures, and assessment, which she
tries to address informally in each class.  She tries to approach lesson
planning with an open mind, not keeping lessons set in a standard form, and
tries as best she can to integrate fun classroom activities so as to encourage
participation.  Occasionally, though, she will block out a set of classes for
discussions, where the first class is in idea building, the next class is in
discussing those ideas, and so forth.  For these classes, the students set the
schedule and progression of material.

Because SZ specifically commented that she's an enthusiastic computer user in
most regards except lesson planning, we asked whether there was anything
specifically that might encourage her to use the computer more for lesson
planning.  She responded that she liked the notebook because she prefers
planning lessons with a tactile and flexible interface; she can't really move
the computer around with her during her teaching or scratch out a line in the
plan or draw an arrow from one point to another.  SZ prefers to plan lessons on
something that's easily editable and can be added to, where she can add notes to
herself while in the classroom as well as at home.

\subsection{AF -- October 12, 2005}
\paragraph{Profile} AF taught Kindergarten through second grade in a bilingual
classroom for 3 years.

\paragraph{Interview}In general, AF planned lessons for the upcoming week.  She
found that planning only one day in advance was completely overwhelming, so
knowing what was going to be happening for the upcoming week was important.  She
noted that it was important to take the district's standards into consideration
during planning.  Before the year began, she decided the order the standards
were going to be achieved in her classroom, and in her school a general theme
was established to help teachers meet these standards.  However, within the
classroom, she generally planned lessons by herself.

AF designed lessons by centering them around the general principles they were
supposed to cover.  She was often given a skeleton plan by the school, and then
set out to find more materials on her own.  AF primarily used on-line resources,
though also got ideas for teaching one group of students as she was teaching
another group.  She commented that it was very difficult to find complete
lessons on-line although it was easy to find specific activities.  It was even
more difficult to find complete lessons that addressed district or state
standards.  She directed us to some websites that she frequented when she
taught.

When AF taught, she searched for lesson plans as well as worksheets and
activities on-line. One challenge she came across was finding materials in both
English and Spanish.  She was able to find lessons she could teach in each
language, but she often had to translate the materials herself, and expressed
that it would be useful to have a template where you could translate materials
from one language to another.  Other resources she frequented were other
teachers, the library, and occasionally she said she simply made the lesson up
from scratch.  She said she would have been happy to share these original lesson
plans, on the assumption that people would share their plans with her.  She
commented that the degree of sharing of lessons would depend on the school
culture, and that there was probably more borrowing and sharing at the
elementary level than at the higher grade levels.

In her classroom, AF often had "centers" that all of the children visited, such
as an art center or a science center, and she was commonly looking for
activities that had to do with themes for each center.  She frequently had to
create new lessons depending on the theme that was being addressed, although she
knew that many other teachers recycled their lesson plans from year to year.
When she changed her lesson plans, she often simply copied and printed off
others' work.  The primary adaptation she made was to make Spanish versions of
the lessons.  She used a lot of ready-made worksheets but often edited others'
worksheets because they didn't follow standards or weren't in both languages.
She also occasionally combined multiple worksheets, taking off unnecessary
questions.

AF commented that it is pretty standard nowadays to have internet available in
the classroom.  The school provided a computer, but did not suggest any websites
to her that might be of interest.  She would have liked to have heard more
information about quality web sites to explore when she was planning lessons.

\section{Literature}
\begin{enumerate}
	\item http://hci.colorado.edu:4242/techcog/uploads/260/Final\_Workspace2.doc

	\item http://preview.dlese.org/

	\item http://swiki.dlese.org/CA-Pilot/1

	\item http://teachingboxes.org/

	\item Khan, H., K. Maull, et al. (2005). \textit{Teaching Boxes: Lessons for
		Digital Library Resource Reuse}. Boulder, Department of Computer
		Science, University of Colorado at Boulder: 12.

	\item Khan, H., K. Maull, et al. (2005). \textit{Customizable Teaching
		Boxes: A Window to Digital Library Resource Reuse}. Boulder, Department
		of Computer Science, University of Colorado at Boulder: 12.

	\item Rosebery, A.S. "What Are We Going to Do Next?": \textit{Lesson
		Planning as a Resource for Teaching}.  In R. Nemirovsky Rosebery, S.
		Ann, J. Solomon, and B. Warren (Eds). (2005), \textit{Everyday matters
		in science and mathematics: Studies of complex classroom events} (pp.
		299-327). Mahwah, NJ, US: Lawrence Erlbaum Associates.

\end{enumerate}



\section{Design Priorities \& Issues}
\subsection{Priorities}

The following items we have identified as being priorities for our design of
the Teaching Boxes interface:

\begin{itemize}
\item Users need to be able to upload individual parts of a lesson.  This may
      include activities, important concepts, external documents, or any other
      pieces that make up a lesson.

\item Along with the ability to upload individual parts, the system also needs
      to be able to separate out these pieces in the lesson.

\item Besides worksheets and concepts, lessons quite often have activities
      involved.  Our system needs to be able to handle these and be able
      to manage multiple activities per lesson.

\item The users of the system will be teachers, people who are not experts
      with computers.  As such, the system needs to be as user-friendly as possible.

\item One of the requirements that our sponsors mentioned is that the system
      needs to be non-linear in nature --- Teachers should be able to navigate
      to any lesson, regardless of their order in a unit.

\item Teaching boxes, once created, need to be editable.

\item Lessons may include outside resources, such as web links.  The system
      needs to handle these in a consistent and friendly manner.

\item Frequently, other electronic materials (such as word documents) accompany
      a lesson.  The system needs to be able to handle these files in a manner
      similar to attaching files to an email message.

\item Some activities require materials, such as glue.  The system should allow
      for the construction of a materials list for lessons and activities.

\item Some lessons take longer than others to teach.  As such, an interface for
      providing time estimates should be supported.

\item One of the most useful things to teachers is the ability to annotate their
      lessons after they've been taught.  As such, the system needs to be able
      to provide this capability.

\item Many times teachers will not want to use a box as-is, but will want to
      take elements from several boxes and create their own.  This dictates the
      need to have both publicly available boxes, as well as private boxes that
      teachers can hold for their own uses.
\end{itemize}

\subsection{Issues}

The following items were identified as potential issues in the design of our
interface:

\begin{itemize}
\item How should the system treat activities?  Possible options might include
      treating all things as ``lessons'' and providing all the fields needed
      for either.  However, this can be wasteful and may show the users a
      confusing amount of information.  Alternatively, lessons and activities
      can be separated out with each having its own set of dedicated fields.

\item Another issue is information visibility.  Creating a teaching box can be
      a fairly complicated task, and it is important that we don't show the
      user too much information at once.  This maintains a clean and minimalistic
      view and avoids overwhelming the users.

\item Many teachers make use not only of on-line resources, but also books,
      textbooks, magazines, and other hard-copy resources.  An issue with our
      system is how to handle these resources, since they obviously cannot be
      uploaded in.
\end{itemize}

\section{Abstract User Profile}
From the interviews we've conducted so far and from our own analysis, we have
been able to come up with an abstract profile describing projected users of this
system.

The user is a teacher of all levels from Kindergarten to $12^{th}$ grade,
focusing in the areas of Math or Science. They should be familiar enough with
computers to do things such as sending email, browsing the Internet, and using a
word processor, but don't necessarily need to be highly experienced users. They
need not be familiar with the system.

\section{Tasks}
Several user tasks have come out of our analysis.  They are as follows.

\begin{enumerate}
The following user tasks are intended to exemplify the uses of the system and the different types of users who might find the system useful. The tasks were derived from our user interviews.
1. Mrs. Anderson, a middle school English teacher for over 30 years, has a completed lesson plan that has proven to work well over time which involves acting out a scene in a play her class is reading. She has decided that she would like to share this lesson with others so that they may make use of her experiences,
information, and plans.  She's familiar with paper teaching boxes in her school, and has only begun using a computer within the last year.
2. Mr. Multon is a second grade physical science teacher for an elementary school. He has a lesson for an ``earthquakes'' unit, but needs to plan an activity to accompany it.  He also has an abundance of play-doh at his disposal.
3. Ms. Frizzle just started teaching science this year to high schoolers, and recently taught a lesson she obtained from another teacher. She found some issues with teaching it.  For example, sometimes the chemicals that were called for had unusual reactions, and the classroom needed to be evacuated because of the fumes.  She would like to make notes on the lesson about her findings for next time.
4. Mr. Johnson teaches Spanish to advanced high school students, and has done so for five years. He is planning a lesson on food names to be taught the following week. He has already assembled much of the information he needs, and feels comfortable with his plan.  Two days later he gets a worksheet from a fellow teacher on ordering at a restaurant that he thinks would add to his lesson very nicely. He wants to incorporate it into his plan.
5. George works for the Department of Education and is developing a new content standard to be taught at the 5th grade level. He wants to design a sample lesson demonstrating how the standard could be fulfilled, and provide it to school districts to use.
\end{enumerate}



\section{Testing Concerns}
Although we anticipate final prototype testing to conclude without issue, there are
some concerns that have been raised.  First is the ever-present worry about not
having enough users to test with.  Even now, we have seen that some possible
interview candidates dropped out at the last minute.

To fully test the system, we will need to guide teachers through actions such as
creating new boxes, modifying existing ones, creating activities, uploading external
files, and creating links to external material.  These actions should cover the
tasks we have identified above, and test the complete robustness of the system.
However, since there are a lot of actions involved, it may be very difficult to
test all of them within a short amount of time.

Along the same lines, it will not be possible for a teacher to use the system to
create a new lesson plan completely from scratch, as this can often times be a
long process.  A possible work-around is to have them come prepared with a plan
they have already completed, and have them simply enter the plan into the system.

For the modification testing, we will most likely need to provide a sample box or
two for teachers to use.  This means that either we'll need to use the existing
boxes on the current Teaching Boxes web site, or create some of our own.  We could
alternatively use the boxes that our testers create on other teachers.

Another concern is that we will need a computer at the testing session with the
appropriate software installed.  This is a concern not because of the ability
to acquire the computer, but because the teachers will not be able to use their
own computers, possibly adding an element of discomfort into the mix.

\end{document}

% vim: ts=4:sw=4
