%\documentclass[letterpaper, twocolumn, 10pt]{article}
\documentclass[letterpaper]{article}
\setlength{\parskip}{5pt}

\title{Test Driven Development of Embedded Systems Using Existing Software Test Infrastructure}
\author{Micah Dowty \\ University of Colorado at Boulder \\ micah@navi.cx}
\date{March 26, 2004}

\begin{document}

\maketitle
\begin{abstract}
Traditional software development methodologies mirror the development cycles used in other
forms of engineering. Specifications are gathered, software is designed, the design is
`constructed` into a finished software product, then the finished product is tested.
New methodologies such as Extreme Programming are improving on this by detecting problems
early and flexing under changing requirements.

We focus on applying Extreme
Programming's test driven development to embedded systems featuring custom hardware and
software designs. An inexpensive and effective method for testing embedded systems using existing
software test frameworks and methodology is developed, applied, and evaluated.
\end{abstract}

\section{Introduction}

Traditionally, software development mirrors the engineering practices common in older
disciplines such as mechanical or electrical engineering. Specifications were gathered
from the customer, the specifications were converted to a design, then the design was
`constructed' into a working piece of software. In the 1980s, software engineers began
to see the flaws in this method: requirements change often and the `construction' process
isn't nearly as predictable as the construction of a highway or an amplifier.

Several new `agile' software methodologies were invented to try to combat this problem.
Many of these methods focus on acknowledging that software development is an unpredictable
process that requires creativity throughout. Extreme Programming is one such methodology.

Traditional software development methodologies divide a project into specification, design, implementation,
and testing. A fully working product is not available until the very end, so there's a good
chance it's not at all what the customer really needs, and serious problems will be hidden
until testing is underway.

Extreme Programming, however, is based largely on test driven development. A developer
writes failing test cases for some bit of necessary functionality, then writes, debugs,
and refactors as necessary until 100\%
of the test cases pass. There always exists a known-working copy of the software with a subset
of the final features, and programmers are free to refactor as necessary without creating
hidden bugs.

The technique developed in this paper applies some of the lessons learned in software development
to the development of hardware and software for embedded systems. We will refer only to embedded
systems throughout the rest of this paper, however the same techniques should be equally applicable to
other `soft' hardware systems, such as those based on Field Programmable Gate Arrays.

\section{The Problem}

Embedded systems are usually built from a diverse set of custom components. This includes
conventional software, firmware for very resource-constrained microcontrollers, microcode,
FPGA or ASIC designs, and board-level designs.

Software on medium to large platforms may be tested using a conventional test framework,
but the interface to other system components must be simulated. Software on very small
CPUs or for devices with uncommon architectures often falls outside the capabilities of
existing test frameworks. FPGA, ASIC, and board-level designs can be simulated using various
software tools, but in isolation from any other system components. Finished PCBs can be tested
using specialized electrical testing equipment, but this can be expensive and hard to use
during the development cycle.

These conventional tools have their advantages. If, for example, a problem is known
to exist in the microcontroller firmware, the firmware can be debugged using a simulator.
These tools fall short of a complete solution, however, in some of the same areas where traditional
software development methodologies fail. Integration is performed infrequently, problems are
hidden until late in development, and developers have little reassurance that the system will
still function after refactoring a design, fixing bugs, or adding features.

This problem is exacerbated by the fact that proper test equipment is often quite expensive,
so these traditional embedded system development and test techniques can be impossible for
hobbyists or projects on a very low budget. In these environments, integration testing is
usually performed manually and only inexpensive software simulator tools are available.

What's needed is a test and development strategy that is easy and cheap to apply, detects
problems early, and easily adapts to changing requirements.

\section{Test Driven Development}

To solve some of these problems, we will apply Extreme Programming's test driven development methods
to embedded systems with both custom hardware and custom software. Specialized simulator and
test systems are not abandoned completely-- when available, they are still useful for tracking down
a known problem. This solution focuses, however, on incrementally developing and testing a fully
integrated system inexpensively using widely available hardware and software.

To make use of existing software testing infrastructure and pose the problem in a simpler way,
we encapsulate the embedded system under test in a software layer. Once all the required
functionality of the embedded system is represented by this software layer, the software layer
can be tested using any compatible off-the-shelf test framework.

This brings several of the same advantages to embedded system development that test driven development
brings to software development. The system is working and fully integrated quickly, but without all
features. Unit tests reassure the developers that a system is working properly even after extensive
refactoring. Test driven development even has advantages unique to embedded
systems: software rarely stops working for no apparent reason, but a comprehensive and easy to use
test suite for hardware will detect component failures, loose connections, or improper setup
quickly and easily.

There are several concerns in using software test infrastructure for hardware, however. The software
encapsulation must preserve all of the hardware's likely failure modes, and the test suite must
account for these extra failure modes. For example, when testing software it is safe to assume
than assigning a value to a variable will always succeed. When testing embedded systems, this
may not be so. Timing errors, loose connections, or faulty components could cause the write operation
to fail completely or partially.

\section{Case Study}

Using the above method, test driven development was used to create the `Critical Decoder' board for
the Colorado Space Grant Consortium's Citizen Explorer satellite. The board in question was started
late in the satellite's development to replace a faulty earlier design.
The previous board had numerous
problems that were not discovered until a late integration phase. No conventional electrical test hardware
was available, and no unit testing framework was available for the board's CPU, so testing was performed
manually and infrequently. To avoid the problems faced by the earlier design, a test driven development
strategy was adopted for the new one.

The Critical Decoder board is responsible for several low-level functions of the satellite that must
remain under control even if the main flight computer experiences problems. This includes controlling
power to the flight computer and other microcontrollers, routing commands between the satellite's RS-485
network and the radio when the flight computer isn't available, reading several analog sensor values,
monitoring deployment status, and controlling radio transmitter power.

\section{Test Fixture}

Encapsulating the Critical Decoder as a software component required interfacing the board's serial, analog,
and digital I/O signals to the computer that would be running the test suite. This would require, on the
computer's side, a total of two RS-232 serial ports, one RS-485 serial port, one synchronous serial receiver,
17 analog outputs, 5 digital inputs, and 2 digital outputs. Ideally this interface would be accomplished
using an off-the-shelf data acquisition and control board capable of precisely recording and generating
test signals.

On the project's limited budget, however, a simpler solution was needed. For this we designed the
RCPOD,\footnote{Source code and schematics are available at http://navi.cx/svn/misc/trunk/rcpod}
a simple USB-attached acquisition and control board. The RCPOD includes built-in support for RS-485
and 8 analog input channels with 8-bit resolution. Any of the 30 I/O pins not used for RS-485 or analog
input can be used as general purpose bidirectional digital I/O. The RCPOD offers far more I/O for the price
than commercial solutions, but its USB 1.1 low-speed interface is much slower than the PCI or full-speed USB
interfaces used by commercial data acquisition and control systems. This is only a minor disadvantage, as
most of the Critical Decoder's I/O is
very low-speed, and the running time of the test suite isn't particularly important. It does raise problems
when using the RCPOD with synchronous serial signals or short pulses, as explained later. The RCPOD only
provides one RS-485 serial port, so the two required RS-232 ports were supplied by off-the-shelf USB to RS-232 adaptors.

The RCPOD provides a Python\footnote{An easy to use interpreted object-oriented programming language, see http://python.org}
interface to its I/O functionality. I/O pins are represented as objects, with methods to test their state, change
their state, and change their direction. To create a software encapsulation of the Critical Decoder, we build on this
with a Python package for the satellite's command protocol, used to communicate with the Critical Decoder over
its RS-485 and RS-232 interfaces. Commands implemented by the Critical Decoder firmware
are represented as objects in this package. These objects can be called to carry out the corresponding command.
Parameters are automatically serialized into the proper packets, and the response is automatically
converted to the most usable representation possible. Errors are reported by raising exceptions. To
make the test suite itself as simple and readable as possible, it is important for this high-level interface package to be
intuitive.

To build the test suite itself, the RCPOD's Python interface and the higher-level command package are
used together to form a software encapsulation of the Critical Decoder's hardware. Test cases are written
using PyUnit,\footnote{http://pyunit.sourceforge.net} the standard test framework for Python. PyUnit is based
on the proven architecture of JUnit, the standard test framework for Java written by Erich Gamma and Kent Beck.
This illustrates another advantage of encapsulating the entire embedded system as a software component: even
though the Critical Decoder's firmware is written in assembly language for a processor too tiny to host a
test framework, the software encapsulation and test suite can be written in any convenient language.

\section{Development}

Following the test driven design methodology as closely as practical, test cases were added before or shortly
after implementing the hardware or software they test. Much of the Critical Decoder's functionality can be tested by issuing
a series of commands and validating the responses. Writing these test cases can be approached in the same way
as writing test cases for a pure software system: the test case may report a failure in the presence of working
software due to a hardware error, but there is no likely way that the test case could report success in the
presence of faulty software and faulty hardware.

Other test cases, however, must focus on the hardware and on interconnection between components. These test
cases must be designed carefully in order to cover hardware's unique failure modes. Consider a simplified
example in which the system under test is a digital buffer. The test case must ensure that one
digital input on the data acquisition and control board always receives the value being transmitted to a
digital output. A first attempt at such a test case might set the output to zero, test that the input is
zero, set the output to one, then test that the input is one. This test case however
could not reliably detect a faulty N-channel transistor on a CMOS buffer. A faulty buffer of this sort would
drive a one properly, but its output would be high-impedance when
its input is zero. Due to capacitance in the circuit, the value the input would see when the output is low would
depend on initial conditions not measurable by the test suite. A more robust test case for this wire would try
sending a zero, a one, then another zero. If the output pin has failed in the way described, the first
received bit would be undefined but the next two bits would both be one, due to capacitance in the input pin.

Digital inputs and outputs were tested using alternating bit patterns as described above. Additionally, some
tests were run first with nearby I/O lines set to zero and repeated with nearby I/O lines set to one, to make it
possible to detect short circuits between pins. Ideally every combination of outputs would be tested, but considering
the low-speed RS-485 network all commands must pass through, this would be impractical.
The Critical Decoder's watchdog output, analog inputs, and synchronous
serial output posed more of a challenge due to limitations of our low-budget RCPOD hardware.

Most of the Critical Decoder's outputs change slowly and only as a response to some command or timer. It has
an external watchdog output, however, that emits pulses when PPP packets are detected on the radio's serial port.
The Critical Decoder includes a pulse stretcher in software that emits a 13 millisecond pulse when a PPP
packet is detected, and ensures that there is at least 13 milliseconds of space between pulses. Instead of trying
to accurately sample the pulse shape, the output is sampled repeatedly to estimate its duty cycle. When no PPP
packets are present the duty cycle should be 0\%, and when PPP packets are being sent continuously it should
be 50\%. The test case allows a margin of error that should be set according to the expected duty cycle sampling
accuracy.

The Critical Decoder's analog inputs pose another problem. Ideally we would test its 17 analog inputs using 17
analog outputs on the RCPOD, but the RCPOD has no analog output capability. While it does require a full D/A converter
to characterize the linearity of the Critical Decoder's A/D converter, corner cases can be tested using much
simpler hardware. The test fixture for the 17 analog inputs uses 6 of the RCPOD's digital outputs. Pairs of
digital outputs are connected to resistor voltage dividers, yielding three analog outputs capable
of generating approximately 0, 2.5, or 5 volts. These three analog outputs are then repeated across all 17
analog inputs. This gives a dramatic reduction in test fixture resources required, but still covers all
common failure modes including shorts or disconnects on the inputs themselves or on the Critical Decoder's
analog multiplexer's address inputs. The repeating pattern of three outputs never ends cleanly on a power
of two boundary, so a problem with a multiplexer address line would always `fold' the inputs in a way that
would change the pattern.

The Critical Decoder includes one more output that changes too quickly for the RCPOD to sample, a synchronous
serial port used to set the value of a D/A converter that changes the radio's transmit power. The original plan
was to connect one of the RCPOD's analog inputs to the output of this D/A converter, but the converter itself
is in the satellite's radio rather than in the Critical Decoder, and no spares were available. In place of the
D/A converter, a suitable latched shift register circuit was attached to the serial port. The parallel output
of this circuit will always match the value the D/A converter would have been set to. For economy of I/O pins,
only bits representing corner cases were connected to the RCPOD's inputs. Of the 12 bits available, only bits
0, 7, 8, and 11 were used. This would be enough to ensure the bit pattern was not rotated or truncated, as would
be expected if the serial output code failed.

\section{Conclusions}

Test driven development has been extremely successful for the Critical Decoder project. As the rest of the
satellite was being tested during the development of the board, it was helpful to always have a working
prototype with a subset of the final functionality. The test suite offered continuous reassurance that the
Critical Decoder was functioning properly, which was especially important when it was being used as
test equipment for other components in the satellite. Most importantly though, using test driven development
ensured that test procedures were always up to date, something that was foreign to us in dealing with the
old design.

Test driven development makes an effective and efficient development strategy for embedded systems. For
projects on a low budget, good testing can be had using inexpensive data acquisition
and control systems and free test frameworks rather than expensive specialized equipment.

Test driven
development applied to embedded systems yields many of the same benefits it does in software development,
with a few unexpected ones. In prototyping the Critical Decoder, the test suite would immediately notice
a broken wire. This increased confidence when making changes to a breadboard covered in wires, as there
was no risk in creating hidden problems when rearranging components or transporting the prototype.

\section{Future Work}

Test driven development using the software encapsulation method is best applied with a high quality
data acquisition and control system, but for those on a budget it should be possible to create an enhanced
RCPOD or RCPOD-like board that eliminates the problems encountered using it to test the Critical Decoder.

The specific problems encountered could be fixed by adding pulse-measuring functionality and synchronous
serial to the RCPOD's firmware. A more general solution would be to create a way to compile timing-critical
test cases into a form that could execute directly on the RCPOD. The development of an inexpensive test
system with this functionality could make test driven development of embedded systems much more common.

In the Critical Decoder project, a breadboarded interface between the Critical Decoder and the RCPOD had
to be constructed. Ideally a test interface would be equipped with a large number of all-purpose I/O pins
and simple physical adaptors to connect it to different types of connectors. All electrical routing could
be done automatically using multiplexing or an FPGA. This would decrease setup time and increase longevity
of a test fixture by eliminating custom hardware.

\end{document}