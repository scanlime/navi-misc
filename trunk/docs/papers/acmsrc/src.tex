%\documentclass[letterpaper, 11pt]{article}
\documentclass[letterpaper, twocolumn, 11pt]{article}
\setlength{\parskip}{5pt}

\title{Real-Time Histogram Imaging}
\author{
	David Trowbridge \\ University of Colorado at Boulder \\ trowbrds@cs.colorado.edu
\and
	Micah Dowty \\ University of Colorado at Boulder \\ micah@navi.cx
}
\date{March 28, 2005}

% abstract?

\begin{document}
\maketitle

\section{Problem and Motivation}
There are many areas of science and mathematics where a density field is plotted in order
to visualize a system.  Such plots are especially common in the study of nonlinear dynamics
where iterative systems can exhibit many different kinds of behavior.  Traditionally, these
systems have been visualized using very simple imaging techniques which allow only a few
thousand points to be shown.

In many cases, a few black points on a white background is sufficient as a visualization
of such systems.  However, because dynamical systems can exhibit such strange structure,
it can be deceiving since only a small part of a trajectory can be shown in a single image.
When a statement about a dynamical system is made, that statement refers to the long-term
behavior of the system in a particular state, rather than only a short piece of it.  In such
cases, an imaging technique capable of displaying millions of points is required.

\section{Background}
When plotting points of zero size, each pixel is effectively a bucket which collects samples.
Traditionally, a single sample completely saturates a pixel, which means that the bucket
can be represented by a single bit.  The problem with this approach is that when a pixel is
hit more than once, information is lost.  Instead, if we treat the image region as a variable
depth, two-dimensional histogram we can capture all the information contained in a long
computation run.

Instead of using just a single bit to represent a pixel, we can use full 32-bit (or larger,
if necessary) integers.  The result of treating an image in this way is what is known as
high dynamic range imaging (HDRI).  Most images on a computer are represented by a single
byte per channel.  In a greyscale image, this means that 256 different intensity values
can be represented.  In the natural world, we can observe luminosity ranges on the order of
$10^8:1$.  Clearly 8 bits is not sufficient to represent this, so recently a lot of work
and research has been put into storing images using floating- and fixed-point logarithmic
representation.

Unfortunately, there is currently very little capability to display and print these
representations.  To present such an image to a person, a procedure known as
\emph{tone-mapping} is performed.  This process consists of choosing a displayable color for
a particular value in the image.  There are several different operators for this, which
allows the user some control over the final image via parameters such as exposure, gamma
correction and color interpolation.

\section{Related Work}
Tone mapping of high dynamic range images has been studied extensively, primarily in the
context of digital photography.  A comprehensive survey of tone-mapping techniques can be
found in \cite{kd}.  Some particularly influential tone-mapping operators are discussed
in \cite{jt}, \cite{jtjhbg} and \cite{gw}.

Histogram-based imaging techniques have been used previously in the rendering of so-called
``fractal flames.''  These are discussed in Scott Draves' informal paper describing the
internals of the \emph{Electric Sheep} screen-saver software\cite{sd}.

\section{Uniqueness of the Approach}
\subsection{Histogram Imaging in Real Time}
Because total histogram density is maintained during the computation run, the actual
tone-mapping step can be performed at any time.  This allows a low quality image to
be produced almost immediately, while further iterations merely add more detail.  Like
other techniques which progressively refine an image, this method performs particularly
well in interactive applications.  The parameter space of some nonlinear systems can be
explored interactively in a visual way; instead of waiting for an entire computation run
to complete, frames can be rendered at preset time intervals, allowing a user to explore
through input parameters until the desired image is seen.  As soon as the parameters
stop changing, the image quality will start to improve.


\section{Results and Contributions}

\begin{thebibliography}{77}
\bibitem{kd} Devlin, Kate: ``A Review of Tone Reproduction Techniques'' {\it Technical Report CSTR-02-005,} Department of Computer Science, University of Bristol, November 2002.
\bibitem{sd} Draves, Scott: ``The Fractal Flame Algorithm.'' Available online, http://flam3.com/flame.pdf
\bibitem{fdjd} Durand, F. and J. Dorsey: ``Interactive Tone Mapping,'' {\it Rendering Techniques 2000: 11th Eurographics Workshop on Rendering,} pp. 219-230, Eurographics, June 2000. ISBN 3-211-83535.
\bibitem{jt} Tumblin, Jack: {\it Three Methods of Detail-Preserving Contrast Reduction for Displayed Images.} Phd thesis, Georgia Institute of Technology, December 1999.
\bibitem{jtjhbg} Tumblin, J., J.K. Hodgins and B.K. Guenter: ``Two Methods for Display of High Contrast Images.'' {\it ACM Transactions on Graphics,} 18(1):56-94, January 1999, ISSN 0730-0301.
\bibitem{gw} Ward, Gregory: ``High dynamic range imaging,'' In {\it Proceedings of the Ninth Colour Imaging Conference,} November 2001.
\end{thebibliography}

\end{document}
