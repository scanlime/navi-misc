\documentclass{acmsiggraph}
\usepackage{mathptmx}
\usepackage{graphicx}

\usepackage{parskip}


\onlineid{0}

\acmformat{print}

\title{Transcendental Mapping of Metric Space}
\author{
  David Trowbridge\thanks{e-mail: trowbrds@cs.colorado.edu}
\and
  Micah Dowty\thanks{e-mail: micah@navi.cx}
}

\keywords{iterated function systems, chaotic maps, high dynamic range, animation}

\begin{document}

\maketitle

% general background - history of IFSes, etc
\section{Introduction}
\copyrightspace
Iterated Function Systems are a celebrated class of dynamical systems in the
computer graphics world. By defining a constrictive set of functions and
recursing, beautiful fractal images can be created. Classically, these
systems use a set of affine transformations, such as Sierpinski's Gasket.
The addition of nonlinear transformations creates the popular ``flame''
fractal. However, most of these techniques still are based fundamentally
on topology-preserving transformation. If this technique is exapanded to
incorporate space-folding transformations, the function systems can be
simplified to a single map, producing visually coherent images.

Chaotic maps can be thought of as a subclass of Iterated Function Systems.
Whereas traditional iterated function systems use a set equation to create
a tree structure, chaotic maps iterate a single equation. This creates a
single reproducable orbit through the space occupied by the attractor. As
the point jumps around it exposes itself onto the image region, leaving
a characteristic picture of the map.

% techniques
% this should talk about the techniques we used - chaotic maps, stochastic
% noise distributions, histogram imaging, HDR processing, density field stuff,
% that crazy crazy color vector, etc
\section{Exposition}
Although our software has been tested with several maps, we focus
on the Peter de Jong map:
\begin{eqnarray*}
  x_{n+1} &= \sin (a y_n) - \cos (b x_n) \\
  y_{n+1} &= \sin (c x_n) - \cos (d y_n)
\end{eqnarray*}
The initial conditions, $x_0$ and $y_0$ are random. For millions or billions
of iterations, $x_n$ and $y_n$ are plotted on a two-dimensional histogram.
Unlike previous techniques, where each $x_n$ and $y_n$ plotted or darkened
a pixel directly, this histogram technique maintains image intensity as
more iterations are performed. As more samples accumulate in the histogram,
the image simply gets more detailed.



blar blar blar blar blar blar blar blar blar blar blar blar blar blar blar
blar blar blar blar blar blar blar blar blar blar blar blar blar blar blar
blar blar blar blar blar blar blar blar blar blar blar blar blar blar blar
blar blar blar blar blar blar blar blar blar blar blar blar blar blar blar
blar blar blar blar blar blar blar blar blar blar blar blar blar blar blar
blar blar blar blar blar blar blar blar blar blar blar blar blar blar blar
blar blar blar blar blar blar blar blar blar blar blar blar blar blar blar
blar blar blar blar blar blar blar blar blar blar blar blar blar blar blar


% possible applications
% hrm...screensavers? textures? it's...art! really!
\paragraph*{}
blar blar blar blar blar blar blar blar blar blar blar blar blar blar blar
blar blar blar blar blar blar blar blar blar blar blar blar blar blar blar
blar blar blar blar blar blar blar blar blar blar blar blar blar blar blar
blar blar blar blar blar blar blar blar blar blar blar blar blar blar blar
blar blar blar blar blar blar blar blar blar blar blar blar blar blar blar
blar blar blar blar blar blar blar blar blar blar blar blar blar blar blar
blar blar blar blar blar blar blar blar blar blar blar blar blar blar blar
blar blar blar blar blar blar blar blar blar blar blar blar blar blar blar

% future work
% spline-based interpolation through parameter space, automatic evasion
% of fixed-points and limit cycles.

\paragraph*{}
blar blar blar blar blar blar blar blar blar blar blar blar blar blar blar
blar blar blar blar blar blar blar blar blar blar blar blar blar blar blar
blar blar blar blar blar blar blar blar blar blar blar blar blar blar blar
blar blar blar blar blar blar blar blar blar blar blar blar blar blar blar
blar blar blar blar blar blar blar blar blar blar blar blar blar blar blar
blar blar blar blar blar blar blar blar blar blar blar blar blar blar blar
blar blar blar blar blar blar blar blar blar blar blar blar blar blar blar
blar blar blar blar blar blar blar blar blar blar blar blar blar blar blar

\section{Conclusion}

\end{document}
