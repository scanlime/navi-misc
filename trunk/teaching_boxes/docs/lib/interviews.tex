\subsection{Participant 1}
\begin{enumerate}
	\item Background
		\begin{enumerate}
			\item High school science teacher

			\item Taught deaf children

			\item 2 years in Boulder County
		\end{enumerate}

	\item Creating a lesson plan
		\begin{enumerate}
			\item Work from curriculum -- identify unit goals

			\item Resources
				\begin{enumerate}
					\item Textbooks

					\item Quantity of resources increases with the newness of
						the material

				\end{enumerate}

			\item Look for activities

			\item Professional development program -- group planning
				\begin{enumerate}
					\item Discuss ``big ideas''

					\item Identify ``essential questions''

					\item Design units individually in a group setting

					\item Design assessments

					\item ``worked well''

				\end{enumerate}

		\end{enumerate}

	\item Structure
		\begin{enumerate}
			\item Begin with ``journalling question''

			\item Hands-on
		\end{enumerate}

	\item Revision process
		\begin{enumerate}
			\item Revise lessons for each class -- tailor to kids

			\item Consists largely of adding to a lesson

			\item Lab methodology
				\begin{enumerate}
					\item Changing equipment

					\item Poorly worded
				\end{enumerate}

		\end{enumerate}

	\item Sharing lessons
		\begin{enumerate}
			\item Shares labs usually, not lessons

			\item Looks for lessons on-line

			\item Always revises borrowed lessons

		\end{enumerate}

	\item What is a ``lesson plan''?
		\begin{enumerate}
			\item Objective

			\item Layout -- schedule, notes to self

			\item Questions

			\item Examples

			\item Wrap-up/conclusion -- a way to provide closure to a
				unit/lesson

		\end{enumerate}

\end{enumerate}

\subsection{Participant 2}
\begin{enumerate}
	\item Background -- Middle school math teacher

	\item Creating a lesson plan
		\begin{enumerate}
			\item Break year into units

			\item Break down units into lessons

			\item Determine context of lesson within unit

			\item From the context, find the lesson goal

			\item List activities

			\item Illustrations / diagrams

			\item Compile all found resources for a lesson

			\item Play around in MS Word (equation editor) -- creating
				worksheets for students, notes for self

		\end{enumerate}

	\item Collaboration
		\begin{enumerate}
			\item Casual

			\item Shared work that was done independently

			\item Modify borrowed material

			\item Write-up activity instructions when the activity is to be
				shared

		\end{enumerate}

	\item Resources
		\begin{enumerate}
			\item Text books

			\item Catalogs of teaching materials

			\item Internet (e.g. mathforum.org)

		\end{enumerate}

	\item Time scale
		\begin{enumerate}
			\item Plan year before it starts -- over the summer

			\item Plan unit before start of unit -- ideally unit preparation
				contains detailed lessons and activities for the unit

			\item Review old lessons before reuse -- sometimes rewriting,
				sometimes annotating

		\end{enumerate}

	\item What is a lesson plan?
		\begin{enumerate}
			\item Context (within unit)

			\item Goal -- various levels of specificity

			\item Sequence of activities

			\item Wrap-up / conclusion

			\item Assessment

		\end{enumerate}

\end{enumerate}

\subsection{Participant 3}
\begin{enumerate}
	\item Background -- 8 years teaching high school English in Illinois

	\item Creating a lesson plan
		\begin{enumerate}
			\item Determine the course goal

			\item Create curriculum map, unit calendars

			\item Plan in detail the evening before the lesson

			\item Plans go in notebook

		\end{enumerate}

	\item Resources
		\begin{enumerate}
			\item ``Steals'' good ideas wherever she finds them

			\item Experience as a teacher and a student

			\item \textit{The English Journal}

		\end{enumerate}

	\item Structure -- Depends on class
		\begin{enumerate}
			\item Different class levels require different levels of structure
				-- Freshman intro course vs. Senior AP course

			\item AP followed a similar framework for each book

		\end{enumerate}

	\item Planning
		\begin{enumerate}
			\item Planning all the time -- solo

			\item Students help to plan discussion days

			\item Borrow and modify plans\ldots
				\begin{enumerate}
					\item From other teachers

					\item From teaching books

				\end{enumerate}

			\item Sometimes trading unit note books

		\end{enumerate}

	\item Revision
		\begin{enumerate}
			\item Much of the work goes on in her mind

			\item Keeps notes in a notebook -- details she needs to remember

			\item Re-uses maps \& calendars, never notebooks

		\end{enumerate}

	\item Notebook goes \textit{everywhere}

	\item What is a lesson plan?
		\begin{enumerate}
			\item In her head\ldots
				\begin{enumerate}
					\item Objective

					\item Material

					\item Assessment

					\item Procedure

				\end{enumerate}

			\item Notebooks
				\begin{enumerate}
					\item Contains details about the lesson or administrivia she
						needs to remember

					\item Amount of detail depends on familiarity with lesson --
						familiar lessons are largely done from memory, new
						lessons are more detailed in the notebook

				\end{enumerate}

		\end{enumerate}

\end{enumerate}


% vim: ts=4:sw=4
