\begin{enumerate}
\item Mrs. Anderson, a middle school English teacher for over 30 years, has a
completed lesson plan that has proven to work well over time which involves
acting out a scene in a play her class is reading. She has decided that she
would like to share this lesson with others so that they may make use of her
experiences, information, and plans.  She's familiar with paper teaching boxes
in her school, and has only begun using a computer within the last year.

\item Mr.  Multon is a second grade physical science teacher for an elementary
school. He has a lesson for an ``earthquakes'' unit, but needs to plan an
activity to accompany it.  He also has an abundance of play-doh at his disposal.

\item Ms. Frizzle just started teaching high school science this year,
and recently taught a lesson she obtained from another teacher. She found some
issues with teaching it.  For example, sometimes the chemicals that were called
for had unusual reactions, and the classroom needed to be evacuated because of
the fumes.  She would like to make notes on the lesson about her findings for
next time.

\item Mr. Johnson teaches Spanish to advanced high school students, and has done
so for five years. He is planning a lesson on food names to be taught the
following week. He has already assembled much of the information he needs, and
feels comfortable with his plan.  Two days later he gets a worksheet from a
fellow teacher on ordering at a restaurant that he thinks would add to his
lesson very nicely. He wants to incorporate it into his plan.

\item George works for the Department of Education and is developing a new
content standard to be taught at the 5$^{th}$ grade level. He wants to design a
sample lesson demonstrating how the standard could be fulfilled, and provide it
to school districts to use.

\end{enumerate}

