We interviewed teachers from a variety of fields with varying levels of
experience, who were trying to reach different age groups, and who had different
classroom challenges, such as teaching in a bilingual classroom and teaching
deaf students.  This approach had the benefit of allowing  us to identify
consistencies in lesson planning clearly, as well as informing us of information
that was more discipline-specific.  Throughout the interviews, a handful of
common themes emerged.  One important point that arose was the need to
explicitly identify the overall question or point to the lesson, and center the
plan around that.  Oftentimes it helped teachers to have a detailed outline of
the day's lesson.  Everyone, however, mentioned how helpful it was to use
activities, laboratories, and similar sorts of things from other teachers, and
this eventually turned into one of our key design points.  Many people mentioned
the importance of adhering to state guidelines and the difficulties of knowing
which lesson plans addressed which guidelines.  Some of the teachers exchanged
lesson plans with others, and some worked together collaboratively on lesson
planning, but the primary emerging theme was that the time and energy necessary
for collaborative work was often lacking in the schools.

When a teacher borrowed another's activities, a very common practice compared to
borrowing an entire lesson, it almost always involved editing other's lesson
plan.  Not a single teacher mentioned borrowing and using a lesson plan
verbatim.  When teachers were asked about how frequently they edited and reused
their own plans, the answer was often that they did, but the ease of editing
one's own lesson plans was a crucial factor.  One problematic factor for this
particular project is that even very computer-savvy users preferred using paper
and pencil to planning lessons for the ease of carrying around and editing
on-the-spot.  Although the availability of this site as a database of lesson
plans would be a rich resource, the important question remains as to whether
teachers will actually use this in a real classroom environment.

\subsection{Individual User Profiles}
\paragraph{KK} KK taught high school science for deaf students.  She taught for
two years in Boulder county.  She now is working on an advanced degree in
education at CU-Boulder.

\paragraph{GF} GF taught middle school mathematics for twelve years, four in Trenton, NJ
and eight here in Boulder.  She did not teach classes last year but did some
tutoring at the same school she taught at, and was not teaching at all this
year. She now is working on an advanced degree in education at CU-Boulder.

\paragraph{SZ} SZ taught high school English for eight years in Illinois.  She
taught a range of classes there from introductory classes for freshmen to
senior-level Advanced Placement classes, and is currently teaching a course for
undergraduates at CU.  She now is working on her PhD in education at CU-Boulder.

\paragraph{AF} AF taught kindergarten through second grade in a Spanish-English
bilingual classroom for three years.  Currently, she's working on her PhD in
cognitive psychology at CU-Boulder.

\paragraph{JR} JR taught Spanish at Cherry Creek Middle School for three years
and high school Spanish for five years. She has taught a variety of levels from
beginning all the to senior-level classes. She now is working on an advanced
degree in Education at CU-Boulder.

\paragraph{BBS} BBS has been teaching for twelve years. She has taught Kindergarten
and first grade, and is now teaching third grade at South Elementary
School in Douglas County.

\paragraph{KW} KW taught high school social studies for thirty-five years in a second
ring Minneapolis suburb. He started teaching in a modular schedule where most of
his classes met three times a week. Later he taught in a seven-period day and finally in
a modular schedule where classes met 90 minutes a day, but for only a semester
rather than a full year.

% vim: ts=4:sw=4
