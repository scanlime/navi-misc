\paragraph{Lesson} Explore various types of plate boundaries in plate tectonics
using playdough to model them. This lesson not only teaches plate tectonics, but
builds elementary scientific writing skills. It is best suited to fifth or sixth
grade students.

For this lesson you'll require a camera (preferably digital) to photograph the
experiments and enough playdough for the entire class, preferably of various
colors. After learning about the three types of plate boundaries -- transform,
divergent, and convergent -- the students will create working examples of these
different boundaries using the playdough. For each boundary they will photograph
their model at different stages as they reenact the motion of the boundary. They
then take these pictures and write a description for each one explaining what
type of boundary it is, what is happening in the picture, and what effect
that has on the Earth's surface (such as creating mountains, volcanoes, etc).
For a slightly more intense focus on scientific writing, this write up can be
done more like a formal lab report.

Generally this activity will take about two to two and a half hours spread out
over 2 days. Plan on about an hour to an hour and a half to create and
photograph the models of the plate boundaries, and then another hour to do the
write up. The write up can also be a homework assignment. It is generally easier
to take more than one day so that there is time to make prints of the
photographs.
