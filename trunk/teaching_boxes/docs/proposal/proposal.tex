\documentclass[11pt,letter]{article}

\usepackage{fancyhdr}
\usepackage{graphicx}
\usepackage{lastpage}
\usepackage[body={6.0in,8.0in}, left=1in, right=1in, top=1in, bottom=1in]{geometry}

\pagestyle{plain}

\title{Teaching Boxes Builder\\Project Proposal}
\author{Deanna Fierman \and Cory Maccarrone \and W. Evan Sheehan}

\begin{document}
\begin{titlepage}
	\maketitle
	\thispagestyle{empty}
\end{titlepage}

\section{Problem}

Teaching boxes are digital collections of lesson plans, activities, and other
resources compiled by teachers. These compilations of materials can be utilized
by other teachers, and the lessons may be modified or added to depending on each
individual teacher's needs. Traditionally, these sets of materials are compiled
in binders or boxes, and commonly become cumbersome to use due to the amount of
material they contain.  In the Teaching Boxes Builder project, materials are
compiled and organized digitally to reduce the clumsiness of compiling large
quantities of teaching material.

Much of the digital Teaching Boxes design work is being done by DLESE (Digital
Library for Earth System Education, http://preview.dlese.org/). DLESE has a
pilot website for the digital teaching boxes found at http://teachingboxes.org/.

One of the primary issues surrounding the design of digital teaching boxes is
ease of creation.  Although a teacher may have a clear lesson plan, making this
lesson and related materials available on-line has thus far been something of a
challenge.  At the present time, the teacher creates the teaching box in
Microsoft Word using a template and then sends Word document to an intermediary.
The intermediary then generates HTML from the Word document and uploads the
teaching box to the web-server. Similarly, modifying a teaching box can not be
done directly. A teacher must send the changes to someone who can propagate the
changes to the teaching box on the website.

We propose to implement a user-friendly, flexible way to allow teachers to
create teaching boxes and modify them directly. Utilizing a template provided on
the website, teachers will be able to create brand new teaching boxes that will
be immediately available to other users. It will be possible to store partially
created lessons for completion at a later time, and to modify existing teaching
boxes. The interface will need to make all the necessary steps in the creation
process clear to new users without bogging down expert users.

\section{References}
\begin{enumerate}
	\item http://hci.colorado.edu:4242/techcog/uploads/260/Final\_Workspace2.doc

	\item http://preview.dlese.org/

	\item http://swiki.dlese.org/CA-Pilot/1

	\item http://teachingboxes.org/

	\item Khan, H., K. Maull, et al. (2005). \textit{Teaching Boxes: Lessons for
		Digital Library Resource Reuse}. Boulder, Department of Computer
		Science, University of Colorado at Boulder: 12.

	\item Khan, H., K. Maull, et al. (2005). \textit{Customizable Teaching
		Boxes: A Window to Digital Library Resource Reuse}. Boulder, Department
		of Computer Science, University of Colorado at Boulder: 12.

	\item Rosebery, A.S. "What Are We Going to Do Next?": \textit{Lesson
		Planning as a Resource for Teaching}.  In R. Nemirovsky Rosebery, S.
		Ann, J. Solomon, and B. Warren (Eds). (2005), \textit{Everyday matters
		in science and mathematics: Studies of complex classroom events} (pp.
		299-327). Mahwah, NJ, US: Lawrence Erlbaum Associates.

\end{enumerate}

In addition, if we have further questions, we will review the sources cited in
the above papers to see if any of them can direct us to an answer

\section{Technical Approach}

Our first step in designing this user interface will be interviewing teachers so
that we will have data from which we can create user tasks. The tasks will be
used throughout the development process to generate and evaluate prototypes. We
will make use of both low- and high-fidelity prototypes to try and discover as
many problems as early as possible in our design. Additionally, we intend to
evaluate the design throughout the process via cognitive walkthroughs. Finally,
we will test the end result with some of the teachers we interviewed for our
user tasks at the beginning of the process.

We want to start with user interviews because we feel it is important to begin
the design process with as much information about the domain that we can gather.
This will give us a solid foundation from which to begin our design. We start
with low-fidelity prototypes and build from there so that we may begin
evaluating our design as soon as possible. In this way we hope to eliminate
problems with the design early so that we may discover as many as we can by the
end of the project. Our final testing is intended not only to provide us with
feedback to fix our design, but also to suggest further work on this project
once our part has ended.

\section{Implementation}
The implementation of the high-fidelity interface will be done primarily in HTML
because the teaching boxes are a web-based resource.  We intend to implement a
robust, dynamic, and easy-to-use interface that will run on any standard
web-browser.  The design may also include the use of JavaScript or another
language such as PHP, depending on the level of dynamic content needed and the
features supported by the underlying web-server.

\section{Evaluation Plan}
We are planning on three different phases of evaluation throughout the life of
the project. First, the low-fidelity and high-Fidelity prototypes will be
constantly evaluated by the team using cognitive walkthroughs. The walkthroughs
will provide us with a means to start improving the design early in the
project's life. Periodic, informal design reviews will also be used to
supplement the cognitive walkthroughs.

The second evaluation phase will occur during the high-fidelity design phase.
In addition to the cognitive walkthroughs performed by the team, the
high-fidelity prototype will be put through a formal cognitive walkthrough by
our classmates in CSCI 4838. This evaluation will provide us with feedback from
people who are already familiar with the evaluation process, but are detached
from the project itself.

Finally, once the interface is complete, we will begin formal user testing with
teachers. The teachers will be asked to use the interface to create a new
teaching box. Feedback from the teachers regarding their experience will be
provided either with notes the teachers make while they are creating the box, or
with a survey the teachers take upon completing the box. From the comments we
receive from this test, we will be able to make our final changes to complete
the project.

\section{Schedule}
\begin{tabular}[!h]{rl}
	\textbf{Date} & \textbf{Task} \\
	\hline
	29 Sept. & Interview questions complete, interviews scheduled \\
	19 Oct.  & Design brief due \\
	20 Oct.  & Low-fi prototype ready for evaluation \\
	27 Oct.  & Low-fi prototype evaluation complete, begin high-fi prototype \\
	10 Nov.  & Evaluation of high-fi prototype \\
	17 Nov.  & Begin final testing \\
	28 Nov.  & First draft of report due \\
	7 Dec.   & Final presentation
\end{tabular}

\section{Individual Work Assignments}
Because of the group's size, most tasks will be handled by all the group
members. Some tasks have been assigned to specific group members, but only to
indicate that those members are primarily responsible for the task. The other
group members will also be working on those tasks.

\hfill
\linebreak
\begin{tabular}[!h]{ll}
	\textbf{Task}         & \textbf{Assigned to} \\
	\hline
	Scheduling interviews & Deanna, Evan \\
	Task definitions      & Everyone \\
	Low-fi prototype      & Everyone \\
	High-fi prototype     & Cory, Evan \\
	High-fi evaluation    & Everyone \\
	Documentation         & Everyone \\
	Final testing         & Everyone
\end{tabular}

\section{Exposures}
Our biggest concern regarding this project is our ability to get enough user
interviews at the start. It could be very difficult to schedule interviews with
teachers who are busy teaching a class and students who are busy taking classes.
We would like to get a good survey of teachers so that we have a strong set of
data from which to build our design. However, if we only get a handful of
interviews, we will have to make do.

Similarly, we are concerned that it will be difficult to get a significant
portion of the class to help us with cognitive walkthroughs on the high-fidelity
prototype. Again, we would like to get a sizable data sample to help us improve
our design. But if we only get a few volunteers -- or worse, none -- it will
have to suffice.

Finally, we do not know what restrictions will be placed on our project by the
web-server that the teaching boxes website is already running on. Though it is
our intention to use JavaScript for the implementation, that may not be possible
on the current web-server. Our implementation could be limited by the resources
available on the web-server.

\end{document}

% vim: ts=4:sw=4
