\documentclass[11pt,letter]{article}

\usepackage{fancyhdr}
\usepackage{graphicx}
\usepackage{lastpage}
\usepackage[body={6.0in,8.0in}, left=1in, right=1in, top=1in, bottom=1in]{geometry}

\pagestyle{plain}

\title{Teaching Boxes Builder\\Project Proposal}
\author{Deanna Fierman \and Cory Maccarrone \and W. Evan Sheehan}

\begin{document}
\begin{titlepage}
	\maketitle
	\thispagestyle{empty}
\end{titlepage}

\section{Problem}

\section{References}

\section{Technical Approach}
Our first step in designing this user interface will be interviewing teachers so
that we will have data from which we can create user tasks. The tasks will be
used throughout the development process to generate and evaluate prototypes. We
will make use of both low- and high-fidelity prototypes to try and discover as
many problems as early as possible in our design. Additionally, we intend to
evaluate the design throughout the process via cognitive walkthroughs. Finally,
we will test the end result with some of the teachers we interviewed for our
user tasks at the beginning of the process.

We want to start with user interviews because we feel it is important to begin
the design process with as much information about the domain that we can gather.
This will give us a solid foundation from which to begin our design. We start
with low-fidelity prototypes and build from there so that we may begin
evaluating our design as soon as possible. In this way we hope to eliminate
problems with the design early so that we may discover as many as we can by the
end of the project. Our final testing is intended not only to provide us with
feedback to fix our design, but also to suggest further work on this project
once our part has ended.

\section{Implementation}

\section{Evaluation Plan}

\section{Schedule}
\begin{tabular}[!h]{rl}
	\textbf{Date} & \textbf{Task} \\
	\hline
	29 Sept. & Interview questions complete, interviews scheduled \\
	19 Oct.  & Design brief due \\
	20 Oct.  & Low-fi prototype ready for evaluation \\
	27 Oct.  & Low-fi prototype evaluation complete, begin high-fi prototype \\
	10 Nov.  & Evaluation of high-fi prototype \\
	17 Nov.  & Begin final testing \\
	28 Nov.  & First draft of report due \\
	7 Dec.   & Final presentation
\end{tabular}

\section{Individual Work Assignments}
Because of the group's size, most tasks will be handled by all the group
members. Some tasks have been assigned to specific group members, but only to
indicate that those members are primarily responsible for the task. The other
group members will also be working on those tasks.

\hfill
\linebreak
\begin{tabular}[!h]{ll}
	\textbf{Task}         & \textbf{Assigned to} \\
	\hline
	Scheduling interviews & Deanna, Evan \\
	Task definitions      & Everyone \\
	Low-fi prototype      & Everyone \\
	High-fi prototype     & Cory, Evan \\
	High-fi evaluation    & Everyone \\
	Documentation         & Everyone \\
	Final testing         & Everyone
\end{tabular}

\section{Exposures}
Our biggest concern regarding this project is our ability to get enough user
interviews at the start. It could be very difficult to schedule interviews with
teachers who are busy teaching a class and students who are busy taking classes.
We would like to get a good survey of teachers so that we have a strong set of
data from which to build our design. However, if we only get a handful of
interviews, we will have to make do.

Similarly, we are concerned that it will be difficult to get a significant
portion of the class to help us with cognitive walkthroughs on the high-fidelity
prototype. Again, we would like to get a sizable data sample to help us improve
our design. But if we only get a few volunteers -- or worse, none -- it will
have to suffice.

Finally, we do not know what restrictions will be placed on our project by the
web-server that the teaching boxes website is already running on. Though it is
our intention to use Javascript for the implementation, that may not be possible
on the current web-server. Our implementation could be limited by the resources
available on the web-server.
\end{document}

% vim: ts=4:sw=4
