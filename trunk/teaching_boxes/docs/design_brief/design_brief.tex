\documentclass[12pt,titlepage]{article}

\usepackage{fancyhdr}
\usepackage{graphicx}
\usepackage{lastpage}
\usepackage[body={6.0in,8.0in}, left=1in, right=1in, top=1in, bottom=1in]{geometry}

\pagestyle{fancy}
\lhead{Teaching Boxes Builder: Design Brief}
\rhead{\today}
\cfoot{\hrule Page\ \thepage\ of\ \pageref{LastPage}}

\title{Teaching Boxes Builder: \\ Design Brief}
\author{Deanna Fierman  \\ \small{fierman@colorado.edu} \and
        Cory Maccarrone \\ \small{Cory.Maccarrone@colorado.edu} \and
		W. Evan Sheehan \\ \small{Wallace.Sheehan@gmail.com}}

\begin{document}
\maketitle

% Blank page after title.
\hfill
\thispagestyle{empty}
\pagebreak
\setcounter{page}{0}

\section{Problem}
The teaching boxes project intends to provide teachers with a digital resource
for lesson plans and materials. Our project's purpose is to create a user
interface for teachers to create and modify lesson plans in this digital
library.

\section{Methods}
The design process for this project begins with user interviews. From the
interviews we generate tasks that are used to create a low-fidelity prototype.
After the low-fidelity prototype is evaluated, we create a high-fidelity
prototype. The high-fidelity prototype is evaluated by people outside of
the development team. These evaluations provide feedback to improve the design
of the final product, which is developed after the evaluation of the
high-fidelity prototype. Finally, once the final product is ready, it is
evaluated by some target users.

\subsection{Progress}
As of October 19, 2005, we have interviewed 4 users. Each interview lasted about
thirty minutes.  Currently, we are trying to schedule two more interviews.
According to our schedule, we should have already finished interviewing and
started on our low-fidelity prototype. We will have to push the due date for the
low-fidelity prototype back a week. And rather than having a week to evaluate
the prototype after it is finished, we can evaluate it while we develop it so
that the low-fi prototype is completed and evaluated by October $27^{th}$. Then
we will be back on schedule.

\subsection{Interview Questions}
\begin{itemize}
	\item How do you create lesson plans?
	\item How do you revise lesson plans, and how frequently?
	\item How long does it usually take you to create a lesson plan?
	\item Where do you get ideas for lesson plans?
	\item Where do you find resources for lesson plans?
	\item Do you work collaboratively with other teachers?
	\item What's your biggest concern about using someone else's lesson plans?
\end{itemize}

\section{Interviews}
\subsection{KK -- October 6, 2005}
\paragraph{Profile} KK taught high school science for deaf students. She taught
for two years in Boulder county.

\subsection{GF -- October 7, 2005}
\paragraph{Profile} GF taught middle school math.

\subsection{SZ -- October 7, 2005}
\paragraph{Profile} SZ taught high school English for eight years in Illinois.
She taught a range of classes there, from Freshman introduction courses to
Senior Advanced Placement (AP) courses.

\subsection{A -- October 12, 2005}
\paragraph{Profile} A taught Kindergarten through second grade in a bi-lingual
classroom for 3 years.

% FIXME - Do we really need this?
\section{Literature}

\section{Design Priorities \& Issues}

\section{Abstract User Profile}

\section{Tasks}

\section{Testing Concerns}

\end{document}

% vim: ts=4:sw=4
