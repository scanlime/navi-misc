\documentclass[12pt,titlepage]{article}

\usepackage{fancyhdr}
\usepackage{graphicx}
\usepackage{lastpage}
\usepackage[body={6.0in,8.0in}, left=1in, right=1in, top=1in, bottom=1in]{geometry}

\newcommand{\blankpage}{\hfill\thispagestyle{empty}\pagebreak\addtocounter{page}{-1}}

\pagestyle{fancy}
\lhead{Teaching Boxes Builder: Design Brief}
\rhead{\today}

\title{Teaching Boxes Builder: \\ Design Brief}
\author{Deanna Fierman  \\ \small{fierman@colorado.edu} \and
        Cory Maccarrone \\ \small{Cory.Maccarrone@colorado.edu} \and
		W. Evan Sheehan \\ \small{Wallace.Sheehan@gmail.com}}

\begin{document}
\maketitle

\blankpage

\pagenumbering{roman}
\tableofcontents
\cfoot{\hrule \thepage}
\pagebreak

\blankpage

\pagenumbering{arabic}
\cfoot{\hrule Page\ \thepage\ of\ \pageref{LastPage}}

\section{Problem}
The teaching boxes project intends to provide teachers with a digital resource
for lesson plans and materials. Our project's purpose is to create a user
interface for teachers to create and modify lesson plans in this digital
library.

\section{Methods}
The design process for this project begins with user interviews. From the
interviews we generate tasks that are used to create a low-fidelity prototype.
After the low-fidelity prototype is evaluated, we create a high-fidelity
prototype. The high-fidelity prototype is evaluated by people outside of
the development team. These evaluations provide feedback to improve the design
of the final product, which is developed after the evaluation of the
high-fidelity prototype. Finally, once the final product is ready, it is
evaluated by some target users.

\subsection{Progress}
As of October 19, 2005, we have interviewed 4 users. Each interview lasted about
thirty minutes.  Currently, we are trying to schedule two more interviews.
According to our schedule, we should have already finished interviewing and
started on our low-fidelity prototype. We will have to push the due date for the
low-fidelity prototype back a week. And rather than having a week to evaluate
the prototype after it is finished, we can evaluate it while we develop it so
that the low-fi prototype is completed and evaluated by October $27^{th}$. Then
we will be back on schedule.

\subsection{Interview Questions}
\begin{itemize}
	\item How do you create lesson plans?
	\item How do you revise lesson plans, and how frequently?
	\item How long does it usually take you to create a lesson plan?
	\item Where do you get ideas for lesson plans?
	\item Where do you find resources for lesson plans?
	\item Do you work collaboratively with other teachers?
	\item What's your biggest concern about using someone else's lesson plans?
\end{itemize}

\section{Interviews}
\subsection{KK -- October 6, 2005}
\paragraph{Profile} KK taught high school science for deaf students. She taught
for two years in Boulder county.

The first KK does when creating a lesson plan is to identify unit goals from the
curriculum. She uses resources like text books to develop each lesson. The
amount she uses the resources depends on how familiar the lesson is to her; more
familiar lessons require fewer resources during development. She also spends
time going over labs, activities, and her own lessons developed previously. KK
looks for ``essential questions'': important questions that should be answered
for the students by the end of the lesson or unit. Her lessons usually start
with a journaling question to start the students thinking about the lesson. As
a science teacher, she did a lot of hands-on exercises in her classes.

KK worked for a time with a group of teachers as part of a professional
development program where they did their lesson planning together. In these
lesson planning sessions they would each work on their own lesson plans
individually, but because they were in a group, they were able to bounce ideas
off of each other and give suggestions. Although she feels that this type of
group planning is not very common, KK felt that this arrangement worked
well.

Because all classes are different, KK revises her lesson plans with each new
class to better suite them. This usually involves adding things to the lesson,
rewriting lab methodology, or updating a lesson for new equipment. She might
change the reading level or questions to better match the class. KK shared labs
with other teachers more often then she shared lesson plans. She also shared
concepts for lessons with other teachers. Sometimes, she would look on-line for
lesson plans, but never used them unmodified.

In her mind, a lesson plan consists of an objective, a layout, questions,
examples, and some form of conclusion. The objective comes from unit goals and
the curriculum. A layout contains notes about the lesson, a schedule, etc. And
the conclusion of the lesson should tie the lesson into the unit somehow.

\subsection{GF -- October 7, 2005}
\paragraph{Profile} GF taught middle school math.

GF begins creating a lesson by breaking down the year into units, and then
breaking down the units. This helps her to identify the context of each lesson
within the unit. Using the context, GF defines a goal for the lesson. She
creates a list of activities. The degree of detail in the activity depends on
the newness of the activity: more familiar activities require less detail.
Because she didn't have to submit formal lesson plans, she just kept a notebook
of her plans.

Most of the collaborative planning she did was casual. GF would discuss her
class with other teachers, and they would share ideas. They would sometimes
share lessons that were developed individually, and make modifications to them.
She always planned her year out before it started, and her units before the
start of each one. She reviewed each of her old lessons before reusing them,
sometimes making changes or annotating them.

GF considers a lesson plan to be a context, a goal, activities, a wrap-up, and
an assessment. The context defines the lesson's relationship to the rest of the
unit. The wrap-up supplies a means to connect the lesson to the rest or the unit
and provide closure for the students.

\subsection{SZ -- October 7, 2005}
\paragraph{Profile} SZ taught high school English for eight years in Illinois.
She taught a range of classes there, from Freshman introduction courses to
Senior Advanced Placement (AP) courses.

\subsection{AF -- October 12, 2005}
\paragraph{Profile} AF taught Kindergarten through second grade in a bi-lingual
classroom for 3 years.

% FIXME - Do we really need this?
\section{Literature}

\section{Design Priorities \& Issues}

\section{Abstract User Profile}

\section{Tasks}

\section{Testing Concerns}

\end{document}

% vim: ts=4:sw=4
