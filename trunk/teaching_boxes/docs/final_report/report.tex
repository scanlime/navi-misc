\documentclass[10pt,letter]{article}

\usepackage{fancyhdr}
\usepackage{graphicx}
\usepackage{lastpage}
\usepackage[body={6.0in,8.0in}, left=1in, right=1in, top=1in, bottom=1in]{geometry}

\newcommand{\blankpage}{\hfill\thispagestyle{empty}\pagebreak\addtocounter{page}{-1}}

\pagestyle{fancy}
\lhead{Teaching Boxes Builder: Final Report}
\rhead{\today}

\title{Teaching Boxes Builder: \\ Final Report}
\author{Deanna Fierman  \\ \small{fierman@colorado.edu} \and
        Cory Maccarrone \\ \small{Cory.Maccarrone@colorado.edu} \and
		W. Evan Sheehan \\ \small{Wallace.Sheehan@gmail.com}}

\begin{document}
\maketitle
\pagebreak
\blankpage

\pagenumbering{roman}
\tableofcontents
\cfoot{\hrule \thepage}
\pagebreak

\blankpage

\pagenumbering{arabic}
\cfoot{\hrule Page\ \thepage\ of\ \pageref{LastPage}}

\section{Problem}
Teaching boxes are digital collections of lesson plans, activities, and other
resources compiled by teachers. These compilations of materials can be utilized
by other teachers, and the lessons may be modified or added to depending on each
individual teacher's needs. Traditionally, these sets of materials are compiled
in binders or boxes, and commonly become cumbersome to use due to the amount of
material they contain.  In the Teaching Boxes Builder project, materials are
compiled and organized digitally to reduce the clumsiness of compiling large
quantities of teaching material.

Much of the digital Teaching Boxes design work is being done by DLESE (Digital
Library for Earth System Education, http://preview.dlese.org/). DLESE has a
pilot website for the digital teaching boxes found at http://teachingboxes.org/.

One of the primary issues surrounding the design of digital teaching boxes is
ease of creation.  Although a teacher may have a clear lesson plan, making this
lesson and related materials available on-line has thus far been something of a
challenge.  At the present time, the teacher creates the teaching box in
Microsoft Word using a template and then sends Word document to an intermediary.
The intermediary then generates HTML from the Word document and uploads the
teaching box to the web-server. Similarly, modifying a teaching box can not be
done directly. A teacher must send the changes to someone who can propagate the
changes to the teaching box on the website.

We propose to implement a user-friendly, flexible way to allow teachers to
create teaching boxes and modify them directly. Utilizing a template provided on
the website, teachers will be able to create brand new teaching boxes that will
be immediately available to other users. It will be possible to store partially
created lessons for completion at a later time, and to modify existing teaching
boxes. The interface will need to make all the necessary steps in the creation
process clear to new users without bogging down expert users.

\section{Related Research}
%FIXME

\section{Approach}
We began by interviewing teachers about lesson planning -- interview questions
can be found in Appendix \ref{interview questions}. The interviews helped us
gather data regarding how teachers create lesson plans and what requirements
they have for digitizing lesson plans. This data was used to create some
low-fidelity prototypes for various aspects of the user interface. We put these
prototypes through cognitive walk-throughs and heuristic evaluations to try and
improve their design before we began any actual coding. After evaluating the
low-fidelity prototype, we created a functioning high-fidelity prototype based
on our low-fidelity mock-ups. The high-fidelity prototype was created using
XHTML, JavaScript, CSS, and AJAX. After this prototype was complete we went back
to our interviewees to have them evaluate it. We had test users perform
think-alouds on this prototype. The feedback gathered from the think-alouds was
then used to improve the high-fidelity prototype and create a finished project.

\section{Interviews}

\section{Abstract User Profile}

\section{Design Priorities and Issues}

\section{Tasks}

\section{Desing Approach}

\section{Results}

\section{Final Design}

\section{User Testing}

\section{Recommendations}

\section{Web Analytics}

\appendix

\section{Interview Questions}
\label{interview questions}
\begin{itemize}
	\item How do you create a lesson plan?

	\item Do you have a structured process, or is it different every time?

	\item How do you find resources?

	\item Do you work collaboratively?
	\begin{itemize}
		\item Do you collaborate during creation?

		\item Do you collaborate after creation?

		\item How does this process work?
	\end{itemize}

	\item How long does it take you to develop a lesson plan?

	\item What is your biggest concern when using someon else's lesson plan?

	\item What do you think you might like about using someone else's lesson
		plan?

	\item Do you know any other teachers who might like to participate?

	\item May we contact you again for later interviews or if we think of any
		further questions?
\end{itemize}

\section{Interview Notes}

\end{document}


% vim: ts=4:sw=4
