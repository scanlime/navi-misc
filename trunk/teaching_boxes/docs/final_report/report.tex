\documentclass[10pt,letter]{article}

\usepackage{fancyhdr}
\usepackage{graphicx}
\usepackage{lastpage}
\usepackage{subfigure}
\usepackage[body={6.0in,8.0in}, left=1in, right=1in, top=1in, bottom=1in]{geometry}

\newcommand{\blankpage}{\hfill\thispagestyle{empty}\pagebreak\addtocounter{page}{-1}}

\pagestyle{fancy}
\lhead{Teaching Boxes Builder: Final Report}
\rhead{\today}

\title{Teaching Boxes Builder: \\ Final Report}
\author{Deanna Fierman  \\ \small{fierman@colorado.edu} \and
        Cory Maccarrone \\ \small{Cory.Maccarrone@colorado.edu} \and
		W. Evan Sheehan \\ \small{Wallace.Sheehan@gmail.com}}

\begin{document}
\bibliographystyle{plain}
\maketitle
\pagebreak
\blankpage

\pagenumbering{roman}
\tableofcontents
\listoffigures
\cfoot{\hrule \thepage}
\pagebreak

\blankpage

\pagenumbering{arabic}
\cfoot{\hrule Page\ \thepage\ of\ \pageref{LastPage}}

\section{Problem}
Teaching boxes are digital collections of lesson plans, activities, and other
resources compiled by teachers. These compilations of materials can be utilized
by other teachers, and the lessons may be modified or added to depending on each
individual teacher's needs. Traditionally, these sets of materials are compiled
in binders or boxes, and commonly become cumbersome to use due to the amount of
material they contain.  In the Teaching Boxes Builder project, materials are
compiled and organized digitally to reduce the clumsiness of compiling large
quantities of teaching material.

Much of the digital Teaching Boxes design work is being done by DLESE (Digital
Library for Earth System Education, http://preview.dlese.org/). DLESE has a
pilot website for the digital teaching boxes\cite{bib:teachingboxes.org}.

One of the primary issues surrounding the design of digital teaching boxes is
ease of creation.  Although a teacher may have a clear lesson plan, making this
lesson and related materials available on-line has thus far been something of a
challenge.  At the present time, the teacher creates the teaching box in
Microsoft Word using a template and then sends Word document to an intermediary.
The intermediary then generates HTML from the Word document and uploads the
teaching box to the web-server. Similarly, modifying a teaching box can not be
done directly. A teacher must send the changes to someone who can propagate the
changes to the teaching box on the website.

We propose to implement a user-friendly, flexible way to allow teachers to
create teaching boxes and modify them directly. Utilizing a template provided on
the website, teachers will be able to create brand new teaching boxes that will
be immediately available to other users. It will be possible to store partially
created lessons for completion at a later time, and to modify existing teaching
boxes. The interface will need to make all the necessary steps in the creation
process clear to new users without bogging down expert users.

\section{Related Research}
%FIXME

\section{Approach}
We began by interviewing teachers about lesson planning -- interview questions
can be found in Appendix \ref{interview questions}, and notes in Appendix
\ref{interview notes}. The interviews helped us gather data regarding how
teachers create lesson plans and what requirements they have for digitizing
lesson plans. This data was used to create some low-fidelity prototypes for
various aspects of the user interface. We put these prototypes through cognitive
walk-throughs and heuristic evaluations to try and improve their design before
we began any actual coding. After evaluating the low-fidelity prototype, we
created a functioning high-fidelity prototype based on our low-fidelity
mock-ups. The high-fidelity prototype was created using XHTML, JavaScript, CSS,
and AJAX. After this prototype was complete we went back to our interviewees to
have them evaluate it. We had test users perform think-alouds on this prototype.
The feedback gathered from the think-alouds was then used to improve the
high-fidelity prototype and create a finished project.

\section{Interviews}

\section{Abstract User Profile}

\section{Design Priorities and Issues}

\section{Tasks}

\section{Design Approach}

\section{Results}

\section{Final Design}

\section{User Testing}

\section{Recommendations}

\section{Web Analytics}

\pagebreak
\appendix

\section{Low-Fidelity Prototype}
\begin{figure}[htb]
	\centering
	\includegraphics[width=0.9\linewidth]{../../low-fi_prototype/button_mouseover}
	\caption{The basic main window for a new digital lesson.}
	\label{fig: main window}
\end{figure}

\begin{figure}[htb]
	\centering
	\includegraphics[width=0.9\linewidth]{../../low-fi_prototype/linked_lessons}
	\caption{Example of linked lessons in a lesson plan.}
	\label{fig: linked lesson}
\end{figure}

\begin{figure}[htb]
	\centering
	\subfigure[Dialog box shown when creating a new lesson.]{
		\label{fig: new dialog}
		\includegraphics[width=0.45\linewidth]{../../low-fi_prototype/new_lesson_dialog}
	}
	\subfigure[Dialog box shown when adding more fields to a lesson.]{
		\label{fig: new field}
		\includegraphics[width=0.45\linewidth]{../../low-fi_prototype/add_field_dialog}
	}
	\caption{Dialog boxes.}
	\label{fig: dialogs}
\end{figure}

\clearpage

\section{Interview Questions}
\label{interview questions}
\begin{itemize}
	\item How do you create a lesson plan?

	\item Do you have a structured process, or is it different every time?

	\item How do you find resources?

	\item Do you work collaboratively?
	\begin{itemize}
		\item Do you collaborate during creation?

		\item Do you collaborate after creation?

		\item How does this process work?
	\end{itemize}

	\item How long does it take you to develop a lesson plan?

	\item What is your biggest concern when using someone else's lesson plan?

	\item What do you think you might like about using someone else's lesson
		plan?

	\item Do you know any other teachers who might like to participate?

	\item May we contact you again for later interviews or if we think of any
		further questions?
\end{itemize}

\section{Interview Notes}
\label{interview notes}
\begingroup
\renewcommand{\labelenumi}{\Roman{enumi}.}
\renewcommand{\labelenumii}{\Alph{enumii}.}
\renewcommand{\labelenumiii}{\arabic{enumiii}.}
\renewcommand{\labelenumiv}{(\alpha{enumiv})}

\subsection{KK -- October 6, 2005}
\begin{enumerate}
	\item Background
		\begin{enumerate}
			\item High school science teacher
			\item Taught deaf children
			\item 2 years in Boulder County
		\end{enumerate}
	\item Creating a lesson plan
		\begin{enumerate}
			\item Work from curriculum -- identify unit goals
			\item Resources
				\begin{enumerate}
					\item Textbooks
					\item Quantity of resources increases with the newness of
						the material
				\end{enumerate}
			\item Look for activities
			\item Professional development program -- group planning
				\begin{enumerate}
					\item Discuss ``big ideas''
					\item Identify ``essential questions''
					\item Design units individually in a group setting
					\item Design assessments
					\item ``worked well''
				\end{enumerate}
		\end{enumerate}
	\item Structure
		\begin{enumerate}
			\item Begin with ``journalling question''
			\item Hands-on
		\end{enumerate}
	\item Revision process
		\begin{enumerate}
			\item Revise lessons for each class -- tailor to kids
			\item Consists largely of adding to a lesson
			\item Lab methodology
				\begin{enumerate}
					\item Changing equipment
					\item Poorly worded
				\end{enumerate}
		\end{enumerate}
	\item Sharing lessons
		\begin{enumerate}
			\item Shares labs usually, not lessons
			\item Looks for lessons on-line
			\item Always revises borrowed lessons
		\end{enumerate}
	\item What is a ``lesson plan''?
		\begin{enumerate}
			\item Objective
			\item Layout -- schedule, notes to self
			\item Questions
			\item Examples
			\item Wrap-up/conclusion -- a way to provide closure to a
				unit/lesson
		\end{enumerate}
\end{enumerate}

\subsection{GF -- October 7, 2005}
\begin{enumerate}
	\item Background -- Middle school math teacher
	\item Creating a lesson plan
		\begin{enumerate}
			\item Break year into units
			\item Break down units into lessons
			\item Determine context of lesson within unit
			\item From the context, find the lesson goal
			\item List activities
			\item Illustrations / diagrams
			\item Compile all found resources for a lesson
			\item Play around in MS Word (equation editor) -- creating
				worksheets for students, notes for self
		\end{enumerate}
	\item Collaboration
		\begin{enumerate}
			\item Casual
			\item Shared work that was done independently
			\item Modify borrowed material
			\item Write-up activity instructions when the activity is to be
				shared
		\end{enumerate}
	\item Resources
		\begin{enumerate}
			\item Text books
			\item Catalogs of teaching materials
			\item Internet (e.g. mathforum.org)
		\end{enumerate}
	\item Time scale
		\begin{enumerate}
			\item Plan year before it starts -- over the summer
			\item Plan unit before start of unit -- ideally unit preparation
				contains detailed lessons and activities for the unit
			\item Review old lessons before reuse -- sometimes rewriting,
				sometimes annotating
		\end{enumerate}
	\item What is a lesson plan?
		\begin{enumerate}
			\item Context (within unit)
			\item Goal -- various levels of specificity
			\item Sequence of activities
			\item Wrap-up / conclusion
			\item Assessment
		\end{enumerate}
\end{enumerate}

\subsection{SZ -- October 7, 2005}
\begin{enumerate}
	\item Background -- 8 years teaching high school English in Illinois
	\item Creating a lesson plan
		\begin{enumerate}
			\item Determine the course goal
			\item Create curriculum map, unit calendars
			\item Plan in detail the evening before the lesson
			\item Plans go in notebook
		\end{enumerate}
	\item Resources
		\begin{enumerate}
			\item ``Steals'' good ideas wherever she finds them
			\item Experience as a teacher and a student
			\item \textit{The English Journal}
		\end{enumerate}
	\item Structure -- Depends on class
		\begin{enumerate}
			\item Different class levels require different levels of structure
				-- Freshman intro course vs. Senior AP course
			\item AP followed a similar framework for each book
		\end{enumerate}
	\item Planning
		\begin{enumerate}
			\item Planning all the time -- solo
			\item Students help to plan discussion days
			\item Borrow and modify plans\ldots
				\begin{enumerate}
					\item From other teachers
					\item From teaching books
				\end{enumerate}
			\item Sometimes trading unit note books
		\end{enumerate}
	\item Revision
		\begin{enumerate}
			\item Much of the work goes on in her mind
			\item Keeps notes in a notebook -- details she needs to remember
			\item Re-uses maps \& calendars, never notebooks
		\end{enumerate}
	\item Notebook goes \textit{everywhere}
	\item What is a lesson plan?
		\begin{enumerate}
			\item In her head\ldots
				\begin{enumerate}
					\item Objective
					\item Material
					\item Assessment
					\item Procedure
				\end{enumerate}
			\item Notebooks
				\begin{enumerate}
					\item Contains details about the lesson or administrivia she
						needs to remember
					\item Amount of detail depends on familiarity with lesson --
						familiar lessons are largely done from memory, new
						lessons are more detailed in the notebook
				\end{enumerate}
		\end{enumerate}
\end{enumerate}

\subsection{AF -- October 12, 2005}
\begin{enumerate}
	\item Background
		\begin{enumerate}
			\item 3 years teaching grade school: 1 year each in Kindergarten,
				first, and second grade
			\item Bi-lingual classes
		\end{enumerate}

	\item Creating a lesson plan
		\begin{enumerate}
			\item Plan for the week -- it's too difficult waiting until the day
				before
			\item Start from standards, working down to curriculum and then a
				theme
			\item Backward design: kids need to learn X, how do you get to X?
			\item Resources
				\begin{enumerate}
					\item minimal resources provided by the school
					\item activities -- centers: worksheets, computers, \ldots
					\item Find resources online (about 60\%)
					\item Translate the resources into Spanish because most
						are in English -- usually the only modification
						she would make
					\item Would like to find more activities online
					\item Most online lessons didn't follow standards of
						her particular district -- would be more useful
						if they did
					\item It is hard to find resources that meet the
						needs of the students, so she would enrich
						materials found online
					\item Used other teachers as resources
					\item Went to the library for resources
					\item As a last resort invented a lesson entirely from
						scratch
				\end{enumerate}
			\item Created new lessons every year because she changed grade
				levels every year
		\end{enumerate}
	\item Collaboration
		\begin{enumerate}
			\item Mostly worked solo on planning -- small groups would form to
				share resources
			\item Willing to share her lessons/activities -- especially with
				people from whom she wanted to borrow
		\end{enumerate}
	\item Needs structure to discourage sloppiness
\end{enumerate}

\subsection{JR -- October 21, 2005}
\begin{enumerate}
	\item Background
	\begin{enumerate}
		\item Comes from a family of teachers
		\item Spanish at Cherry Creek Middle School 3 years
		\item 5 years of high school Spanish
	\end{enumerate}
	\item Lesson planning
	\begin{enumerate}
		\item Uses a big table
		\item lay out all materials -- text book, etc.
		\item Backwards design -- start from end of chapter
		\item Work from goal to requirements for achieving goal
		\item Gets easier as semester goes on
		\item Organize into categories
		\begin{enumerate}
			\item Grids of activities by category
			\item Keeps grids for future use
			\item Adds to grids when reusing
		\end{enumerate}
		\item Activities
		\begin{enumerate}
			\item Affected by level of students -- freshman vs. seniors\ldots
			\item Affected by time of day of the class -- morning vs. afternoon
			\item Geared towards individuals in class
		\end{enumerate}
	\end{enumerate}
	\item Resources
	\begin{enumerate}
		\item Coworkers
		\item Text book and related materials
		\item Internet -- resources only, not materials
	\end{enumerate}
	\item Cotaught a few lessons
	\begin{enumerate}
		\item Combining Spanish and Art
		\item Paired poor Spanish speakers with poor English speakers to help
			them learn to communicate
	\end{enumerate}
	\item Sharing lessons
	\begin{enumerate}
		\item Share rubriks
		\item Alter borrowed lessons
	\end{enumerate}
	\item Degree of teaching experience dictates how detailed a plan needs to be
	\item A lesson plan is a map of interaction with students
\end{enumerate}

\subsection{BBS -- November 27, 2005}
\begin{enumerate}
\item Creating a lesson
	\begin{enumerate}
	\item Start with standards
	\item Design a pre-test to assess where students are in relation to the goal
	\item Design a post-test to assess progress towards goal
		\begin{enumerate}
		\item Can't assume that the students got the lesson
		\item Need an assessment that determine what the students missed, not
			just that they didn't understand the lesson
		\end{enumerate}
	\item Figure out what to do with students who don't reach the goal by the
		end of the lesson and what to do with those who do
	\item Pre-assess before completing the lesson -- item If only a small group
		don't know the information, tailor the lesson to them
	\item 3 different learning styles per lesson
	\item Looks online for activities
		\begin{enumerate}
		\item Usually find overly specific lessons from college/grad students
		\item Preferes webpages run by teachers
		\item Look online before pre-test to get ideas
		\item Thinking of kids in class as she goes through the websites
		\item Download and tweak resources for the class
		\end{enumerate}
	\item Time frame
		\begin{enumerate}
		\item Depends on the amount of experience
		\item Used to take every night and all weekend as a beginning teacher
		\item Create a map for the week over the weekend as a guide for the week
		\item Plans change as soon as the students walk in
		\item Students attitude affects what they are capable of learning
		\item Especially true of Kindergarten and 1$^{st}$ grade
		\end{enumerate}
	\item Lesson structure -- mini-lesson, practice, mini-lesson, practice\ldots
	\end{enumerate}
\item Finding resources
	\begin{enumerate}
	\item Teaching books
	\item Internet
	\item Teammates
	\end{enumerate}
\item Working collaboratively
	\begin{enumerate}
		\item Meets with team to lesson plan
		\item Create a lesson as a group
		\item Process is the same as planning alone
	\end{enumerate}
\item Sharing lessons
	\begin{enumerate}
		\item Tweak borrowed lessons for the current class
		\item Other lessons act as a creative spark
		\item Concerned that it takes a lot of time to write down a lesson for
			sharing
	\end{enumerate}
\item What is a lesson plan
	\begin{enumerate}
	\item Goal (standards) -- what students need to know
	\item How to achieve goal
	\item How to assess progress towards goal and what to do with those who
		achieve the goal and those who don't
	\item Accounting for high-level, low-level, and ESL students
	\item Good teaching takes into account not only the skills to teach, but
		application of the skills learned -- learning how to think
	\item Students have to participate in the planning/learning
	\end{enumerate}
\end{enumerate}

\endgroup
% vim: ts=4:sw=4


\pagebreak
\bibliography{resources}

\end{document}

% vim: ts=4:sw=4
